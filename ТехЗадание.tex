\newsection
\setcounter{figure}{0}\setcounter{table}{0}
\section{Техническое задание}
\subsection{Основание для разработки}

Основой для разработки послужила дипломная работа:
Разработка сайта ``Программно-информационная система для организации работы медицинской лаборатории`` с использованием языка HTML.

При разработке веб-сайта для клинической лаборатории были учтены некоторые ключевые аспекты. Ниже перечислены некоторые из них:

\begin{itemize}
	\item Определить потребности клинической лаборатории: мы проанализировали, какую информацию и услуги лаборатория хочет предоставить пользователям на своем сайте.
	\item Определить удобный и понятный для пользователя дизайн: веб-сайт должен быть простым в навигации для пользователей и предоставлять информацию в ясной и организованной форме.
	\item Информационная безопасность: необходимо обеспечить безопасность и конфиденциальность информации, предоставляемой пациентами.
	\item Актуальная информация: веб-сайт должен содержать актуальную и точную информацию о лаборатории, предлагаемых услугах и результатах анализов.
	\item Интеграция с технологиями: следует рассмотреть возможность интеграции передовых технологий, чтобы сделать работу пользователя более приятной и эффективной, например, онлайн-платформа для записи информации о пациентах, врачах и специальностях для доставки результатов анализов.
	\item Фокус на пользовательском опыте: веб-сайт должен быть разработан с учетом пользовательского опыта и того, как он может облегчить доступ к информации и услугам, предлагаемым клинической лабораторией.
\end{itemize}

\subsection{Цель и назначение разработки}

Цель веб-сайта в клинической лаборатории - предоставить пользователям лаборатории информацию о предоставляемых услугах, а также позволить им получить доступ к результатам анализов и наблюдать за лечением отдельных пациентов.

Поэтому функция веб-сайта в клинической лаборатории заключается в первую очередь в облегчении доступа пользователей к информации и услугам, оказываемым в лаборатории. Например, веб-страница может предоставлять подробную информацию о различных типах предлагаемых анализов, указывать предыдущие диагнозы пациента, которые будут рассматриваться для лечения, новые анализы или новые лекарства, а также предоставлять инструкции для врачей, которые придут на следующую смену пациента в связи с уже проведенными процедурами.

Цель и функция веб-сайта в клинической лаборатории - быть важным инструментом коммуникации между лабораторией и врачами, обеспечивая быстрый и легкий доступ ко всей необходимой информации, а также облегчая заказ информации и получение результатов анализов.

Основной целью данной выпускной квалификационной работы является разработка и внедрение веб-сайта для продвижения информации врачам и пациентам, проходящим лабораторные исследования в клинике ``KLLaboratory''.

С помощью внедрения веб-сайта планируется устранить существующие недостатки ручного письма и избежать расхода бумажных листов. Веб-сайт обеспечит уверенность и безопасность при лечении пациентов.

Целями этой разработки являются:
\begin{itemize}
\item создание администратора и медицинских пользователей;
\item создание истории болезни пациентов;
\item реализация формы для регистрации каждого пациента по специальности;
\item внедрение формы для персональных данных пациента;
\item реализация формы для персональных данных каждого врача.
\end{itemize}

Разработка предназначена для приобретения навыков в проектировании архитектуры веб-приложения для внедрения в клиническую лабораторию ``KLLaboratory''.

\subsection{Требования пользователя к интерфейсу web-сайта}

Использование на сайте четкого и хорошо продуманного интерфейса может обеспечить ряд преимуществ, среди которых:

\begin{itemize}
	\item Улучшение удобства использования: интуитивно понятный, дружественный интерфейс может помочь пользователям чувствовать себя более комфортно при навигации по сайту и поиске необходимой информации. Это может повысить удовлетворенность пользователей и снизить количество отказов.
	\item Повышение уровня удержания пользователей: если пользователи находят сайт простым в использовании и предоставляют необходимую им информацию в ясной и краткой форме, они с большей вероятностью останутся на сайте дольше и станут потенциальными клиентами.
	\item Улучшение рейтинга в поисковых системах: поисковые системы отдают предпочтение сайтам, которые обеспечивают хорошее удобство пользования. Чистый, хорошо продуманный интерфейс может улучшить пользовательский опыт.
	\item Улучшение брендинга: хорошо продуманный и последовательный интерфейс может передать профессиональный и последовательный образ бренда, что может улучшить имидж компании и повысить ее репутацию.
	\item Сокращение времени загрузки: оптимизированный интерфейс может сократить время загрузки страницы, что улучшает пользовательский опыт и снижает процент отказов.
	\item Персонализация пользовательского опыта: хорошо продуманный интерфейс позволяет персонализировать пользовательский опыт, что может повысить удовлетворенность и удержать пользователя.
	\item Улучшенная безопасность: хорошо спроектированный интерфейс может включать меры безопасности и защиты пользователей, такие как шифрование данных и аутентификация пользователя.
	\item Улучшенный анализ и оценка данных: хорошо продуманный интерфейс позволяет собирать и анализировать данные об использовании страницы, что может помочь улучшить пользовательский опыт и оптимизировать страницу для лучшей конверсии.
\end{itemize}

Сайт должен включать в себя:
\begin{itemize}
    \item авторизацию;
    \item навигацию по разделам;
    \item доступы для администратора и исполнителя по заявкам с форм.
\end{itemize}

Требования пользователей отображаются в шаблоне сайта, как показано на рисунке ниже ~\ref{fig:login}.

\begin{figure}
	\centering
	\includegraphics [width=0.8\linewidth]{Login.jpg}
	\caption{Требования пользователей к странице входа в систему}
	\label{fig:login}
\end{figure}

Меню навигации отображается в шаблоне сайта, как показано на следующем рисунке~\ref{fig:menu}.

\begin{figure}
	\centering
	\includegraphics [width=0.2\linewidth]{menu.png}
	\caption{Меню навигации отображается в шаблоне сайта}
	\label{fig:menu}
\end{figure}

\paragraph{Требования к обработке продукции}

Процесс установки и использования систем управления базами данных может быть сложным. Для использования программного обеспечения для управления данными в клинической лаборатории необходимо подключение к Интернету и компьютер с установленным языком программирования Java. Ввод информации в систему осуществляется с помощью клавиатуры и мыши путем заполнения обязательных полей.

Каждый врач имеет уникальное имя пользователя, которое дает доступ ко всем функциям системы. Для обеспечения правильного хранения информации о госпитализации и лечении пациентов важно, чтобы врачи обладали навыками последовательного и четкого набора текста.

Система должна иметь возможность хранить соответствующую информацию о результатах анализов, проведенных пациентам. У каждого врача и каждого пациента в системе может быть зарегистрировано несколько анализов. Кроме того, система должна иметь возможность поиска и запроса конкретной информации о пациентах и результатах анализов.

В целом, системы управления базами данных являются важнейшими инструментами для управления клинической информацией в лаборатории, и хотя их установка и использование могут быть сложными, они необходимы для обеспечения эффективного управления информацией о пациентах.

\subsection{Моделирование вариантов использования}

Для моделирования вариантов использования программного обеспечения клинической лаборатории первым шагом является определение участников системы, например, лаборантов, врачей, пациентов и т.д. Затем можно определить варианты использования для каждого участника, указав, какие операции они должны уметь выполнять. Затем для каждого участника можно определить варианты использования, указав, какие операции они должны иметь возможность выполнять в системе.

Некоторые варианты использования, которые могут быть актуальны для программного обеспечения клинической лаборатории, могут быть следующими:
\begin{enumerate}
\item Регистрация пациентов: позволяет администраторам системы добавлять и вести записи пациентов, включая демографическую и медицинскую информацию.
\item Запись результатов анализов: позволяет лаборантам вводить результаты анализов, выполненных для каждого пациента.
\item Запрос результатов анализов: позволяет врачам получить доступ к результатам анализов, проведенных их пациентам.
\item Формирование отчетов: позволяет врачам формировать отчеты о результатах анализов, чтобы поделиться ими с пациентами или другими врачами.
\end{enumerate}

Важно определить конкретные сценарии использования, которые имеют отношение к работе клинической лаборатории, а затем смоделировать их соответствующим образом. 

Важно отметить, что моделирование вариантов использования является важным шагом в разработке программного обеспечения, поскольку оно позволяет всем заинтересованным сторонам проекта иметь четкое представление о том, как должна функционировать система и какие операции смогут выполнять различные пользователи.

Одним словом, моделирование сценариев использования является важным этапом в процессе разработки программного обеспечения, поскольку оно помогает понять, какими функциональными возможностями должна обладать система и как различные участники взаимодействуют с ней. В случае клинической лаборатории это особенно важно, поскольку это высокоспециализированная и регулируемая рабочая среда, с особыми требованиями к анализу результатов и управлению клиническими системами.

\paragraph{Спецификация и функции программы}

Функция веб-сайта в основном заключается в предоставлении информации или услуг пользователям через Интернет. Веб-сайт позволяет представить общественности информацию о продукте, услуге, организации, компании и т.д. Кроме того, веб-сайт может служить для обмена мультимедийным контентом, взаимодействия с посетителями, продажи товаров или услуг, создания онлайн-сообщества и других возможных функций в зависимости от его цели. Структура, дизайн и содержание веб-сайта основаны на ряде специальных ресурсов, таких как HTML, CSS и JavaScript, а также базы данных и протокол HTTP для его правильного функционирования.

Необходимо ввести точные данные пациента, обязательно заполнив все обязательные поля соответствующими символами. Необязательные поля не имеют особого значения. На страницах каждой специальности можно найти множество врачей и пациентов, что обеспечивает легкий доступ к истории болезни каждого пациента, а также к соответствующим анализам и результатам.

Каждому пациенту и соответствующей медицинской специальности присваивается уникальный идентификационный номер, который нельзя повторить или изменить в соответствующей палате. Этот номер присваивается пациенту или врачу во время первой консультации.

\subsection{Требования к оформлению документации}

Разработка программной документации и программного изделия должна производиться согласно ГОСТ 19.102-77 и ГОСТ 34.601-90. Единая система программной документации.

\subsection{Требования к составу и параметрам технических  средств}

Минимальные требования к веб-сайту клинической лаборатории включают:

\begin{itemize}
	\item Информация о тестах, проводимых в клинической лаборатории.
	\item Доступ к результатам анализов и возможность регистрации новых результатов анализов.
	\item Актуальное и релевантное содержание о пациентах, используемых процедурах, требованиях к лекарствам и т.д.
	\item Защита конфиденциальности пациентов и предоставляемой информации.
	\item Привлекательный и простой в навигации дизайн для обеспечения наилучшего пользовательского опыта.
	\item Интеграция с передовыми технологиями для облегчения доступа и использования услуг, предлагаемых клинической лабораторией.
	\item Обеспечение соответствия веб-сайта соответствующим стандартам качества и безопасности.
\end{itemize}

Для работы клиентского веб-приложения требуется не менее 10 МБ дискового пространства, 64 МБ свободной оперативной памяти, видеокарта с разрешением экрана не менее 800*600, клавиатура и мышь для операционной системы не ниже Windows 98.

\subsection{Требования к информации и совместимости веб-приложения}

Совместимость веб-сайта в клинической лаборатории - это способность правильно функционировать с различными устройствами и веб-браузерами, а также обеспечивать плавный и эффективный просмотр для пользователей, особенно для пациентов.

Сайт совместим с различными веб-браузерами, такими как Chrome, Firefox, Safari, Opera, Internet Explorer и т. д., сайт также адаптируется к различным размерам экрана, от настольных компьютеров до мобильных устройств, таких как телефоны и планшеты.

Сайт быстро загружается и обеспечивает хороший пользовательский опыт, с простой и доступной навигацией и четким и лаконичным содержанием, сайт соответствует стандартам и правилам, относящимся к медицинскому сектору и защите персональных данных пациентов.

Сайт легко доступен и используется врачами различных специальностей и уровней технической подготовки.

Врач будет использовать клавиатуру в качестве метода ввода и компьютерную мышь для выбора.
Обо всех ошибках, допущенных врачом или при вводе данных через Интернет, система будет уведомлять предупреждающим сообщением с указанием места и причины ошибки.
Продукт веб-приложения должен работать на операционных системах Windows с установленными Sun JDK 1.1.8, Microsoft SDK 3.1, IBM JDK 1.1.7B или более поздней версии. Требуется Visual Studio Code.

\subsection{Стадии и этапы разработки}

Выполнение разработки должно включать три стадии: 

\begin{itemize}
	\item техническое задание;
	\item технический проект;
	\item рабочий проект.
\end{itemize}

На стадии "Техническое задание" проводится постановка задачи, разработка требований к веб-приложения, изучение литературы по задаче и оформление документа "Техническое задание".
На стадии "Технический проект" проводится анализ данной предметной области, выяснение структуры программы резидента. В заключение данного этапа оформляется документ "Технический проект".
На стадии "Рабочий проект" проводится разработка схем алгоритмов для функционального модуля, физическое проектирование программного изделия, разработка тестов, тестирование программных модулей. В заключение данного этапа оформляется документ "Рабочий проект".

Процесс разработки веб-сайта клинической лаборатории включает несколько этапов, некоторые из которых включают следующие:

\begin{itemize}
	\item Планирование: Этот этап включает в себя определение целей веб-сайта, определение содержания и структуры веб-сайта и выявление ресурсов, необходимых для его разработки.
	\item Дизайн и верстка: На этом этапе создается визуальный дизайн сайта, определяется расположение элементов, разрабатывается визуальная иерархия и прорабатывается удобство использования сайта в отношении взаимодействия с врачом.
	\item Разработка: На этом этапе кодируются и реализуются необходимые функциональные возможности, чтобы веб-страница выполняла свою задачу, различные формы интегрируются с базой данных, а элементы оптимизируются с точки зрения их доступности.
	\item Тесты: Проводятся различные тесты для выявления ошибок и гарантии качества сайта, планируются тесты производительности и разрабатываются последующие процессы для решения проблем.
	\item Обслуживание и обновление: После запуска сайта необходимо поддерживать его в актуальном состоянии, обеспечивая актуальность функциональных возможностей и содержания.
\end{itemize}

\subsection{Порядок контроля и приемки}

Продукт веб-приложения изделия осуществляется при сдаче документально оформленных этапов разработки и проведении испытаний на основе установленных тестов. Тесты должны быть предоставлены поставщиком и согласованы с заказчиком.

Проверка и приемка веб-сайта клинической лаборатории - это важный процесс, обеспечивающий оптимальное восприятие веб-сайта клиницистами и соответствие стандартам и нормам, установленным в медицинском секторе.

В этом процессе могут быть реализованы различные этапы тестирования для обеспечения функциональности и интерактивности веб-сайта, а также предоставления необходимой информации и удобства использования. Методы оценки также могут быть рассмотрены для обеспечения качества и безопасности сайта.

Некоторые ключевые области, которые необходимо оценить в ходе проверки и приемки веб-сайта клинической лаборатории, могут включать:

\begin{itemize}
	\item Доступность: проверка правильности отображения сайта на различных устройствах и браузерах, таких как ноутбуки, телефоны и планшеты, для обеспечения его доступности на широком спектре устройств.
	\item Удобство использования: оценить удобство использования сайта, чтобы убедиться, что клиницисты легко понимают представленную информацию, а также инструменты, обеспечивающие интуитивную навигацию.
	\item Информационная безопасность: оценка защиты персональных данных и обеспечение безопасного и надлежащего обращения со всей информацией, связанной со здоровьем пациента.
	\item Соответствие нормативным требованиям: убедитесь, что веб-сайт соответствует юридическим, фармацевтическим и научным стандартам и нормам, действующим в медицинском секторе.
	\item В целом, проверка и приемка веб-сайта - это важный процесс, обеспечивающий соответствие ожиданиям пациентов и косвенным ожиданиям в медицинской среде.
\end{itemize}
