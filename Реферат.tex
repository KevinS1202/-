\newpage
\begin{center}
  РЕФЕРАТ
\end{center}

Объем работы составляет 53 страницы. Работа содержит 9 иллюстраций, 5 таблиц, 15 библиографических источников и 40 листов графических материалов. Количество приложений: 2. Графический материал представлен в приложении А. Дизайн сайта, включающий соединение компонентов, представлен в приложении Б.

Список ключевых слов: веб-страница, клиническая лаборатория, пользователь, врач, пациент, специальности, регистрация, пароль, дизайн, облегчить, база данных, выполнено.

Объектом разработки является веб-сайт клинической лаборатории, занимающейся обследованием пациентов, с квалифицированными врачами и регистрацией проведенных процедур и диагнозов.

В процессе создания сайта основные объекты были выделены путем создания информационных блоков, были использованы классы и модули с функциями, обеспечивающими работу с объектами предметной области, а также правильное функционирование веб - сайта, были разработаны разделы, содержащие информацию о врачи, пациенты и специальности.

Целью дипломной квалификационной работы является автоматизация задач, отказ от использования бумажных листов и гибкость при вводе соответствующих данных о пациентах.

При разработке сайта использовался язык программирования HTML с CSS

Разработанный сайт был успешно внедрен в клинической лаборатории.
\newpage
\selectlanguage{english}
\begin{center}ABSTRACT\end{center}
  
The volume of work is 53 pages. The work contains 9 illustrations, 5 tables, 15 bibliographic sources and 40 sheets of graphic materials. Number of applications: 2. The graphic material is presented in Appendix A. The design of the site, including the connection of components, is presented in Appendix B.

Keyword list: web page, clinical laboratory, user, doctor, patient, specialty, registration, password, design, facilitate, database, completed.

The object of the development is the website of a clinical laboratory engaged in the examination of patients, with qualified doctors and registration of procedures and diagnoses performed.

In the process of creating the site, the main objects were allocated by creating information blocks, classes and modules with functions were used to work with the objects of the subject area, as well as the proper functioning of the website, sections containing information about doctors, patients and specialties were developed.

The purpose of the diploma qualification work is to automate tasks, avoid the use of paper sheets and flexibility in entering relevant patient data.

When developing the site, the HTML programming language with CSS was used

The developed website was successfully implemented in the clinical laboratory.
\selectlanguage{russian}
