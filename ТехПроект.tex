\newsection
\setcounter{figure}{0}\setcounter{table}{0}
\section{Технический проект}
\subsection{Общая характеристика организации решения задачи}

Общая организация веб-сайта клинической лаборатории зависит от конкретного содержания, которое вы хотите представить, и цели, которую вы хотите достичь. Ниже приведены некоторые разделы веб-сайта клинической лаборатории:

\begin{itemize}
	\item Главная страница: страница приветствия, на которой есть поля для входа в систему и ввода пароля врача.
	\item Услуги: раздел с перечнем предлагаемых услуг, таких как анализы крови, клинические анализы и другие медицинские услуги.
	\item Формы: раздел, предназначенный для заполнения личных данных пациента, а также записей о лечении, наблюдении, приеме лекарств и анализах пациента с помощью формы с обязательными и необязательными полями.
	\item Результаты: раздел, где врачи могут получить доступ к результатам лабораторных тестов в режиме онлайн в безопасной и эффективной системе.
	\item Таблицы просмотра: включает таблицу просмотра для поиска ранее зарегистрированных пациентов.
\end{itemize}

Клиническая лаборатория уникальна, поэтому разделы рассматриваются исходя из конкретных целей и потребностей. Сайт интуитивно понятен и обеспечивает оптимальную навигацию для врача в поиске актуальной и полезной информации о пациенте.

Необходимо спроектировать и разработать сайт, который должен способствовать продвижению компании на рынке.

Интернет-сайт представляет собой набор взаимосвязанных электронных страниц, которые сгруппированы по разделам, содержащие текстовую, графическую, а также мультимедийную информацию (изображения, видеоролики и пр.). Сайт располагается в Интернете по определенному адресу – доменному имени сайта в виде www.имя\_сайта.ru. Каждая страница web-сайта – это текстовый документ, написанный на языке программирования (HTML, CSS, JavaScript и т.д.).

\subsection{Обоснование выбора технологии проектирования}

В настоящее время длительные процессы можно решить с помощью собственного программного обеспечения, поскольку хорошая поддержка мультимедиа HTML способна легко интегрировать мультимедийные элементы, такие как изображения, видео и аудио, в веб-страницу, что выгодно, когда вы хотите обеспечить разнообразие в контенте.

Гибкость дизайна HTML как языка разметки позволяет дизайнерам создавать индивидуальные стили и макеты страниц, предоставляя разработчикам полный контроль над визуальным оформлением веб-страниц.


\subsubsection{Описание используемых технологий и языков программирования}

HTML (Hypertext Markup Language) - одна из самых фундаментальных технологий, используемых в веб-программировании. HTML используется для определения структуры и содержания веб-страницы, т.е. элементов, составляющих веб-страницу, и их иерархической организации.

В дополнение к HTML существуют и другие технологии, такие как CSS (каскадные таблицы стилей), которые позволяют определять стили и визуальный формат веб-страницы. Существует также JavaScript, третья фундаментальная технология, которая используется для придания страницам интерактивности и динамичности.

HTML был разработан в начале 1990-х годов и с течением времени развивался, включая новые функции и возможности. Совет Всемирной паутины (W3C) является организацией, ответственной за разработку и поддержание стандартов HTML.

Навыки, необходимые для изучения HTML, включают понимание тегов, базовой структуры веб-страницы, атрибутов, ссылок, форм, а также использование изображений и видео.

В целом, HTML - это важный язык разметки, используемый для разработки современных веб-страниц и являющийся фундаментальным компонентом веб-программирования.

Использование HTML на веб-странице в клинической лаборатории - обычное дело, поскольку HTML - это язык, используемый для создания веб-страниц. HTML, что расшифровывается как HyperText Markup Language, - это язык разметки, который используется для создания структурированного веб-контента. Использование HTML позволяет легко создать организованную и удобную веб-страницу для пациентов.

HTML-теги позволяют организовать содержимое веб-страницы в различные разделы, такие как заголовки, абзацы, таблицы, изображения, формы и ссылки. Клинические лаборатории могут использовать эти теги, чтобы сделать свою информацию легко читаемой и доступной для пациентов.

Дизайн и внешний вид веб-сайта также должны соответствовать клиническому имиджу лаборатории. Поэтому используйте соответствующую цветовую схему и убедитесь, что содержание представлено профессионально и понятно.

Использование HTML на сайте клинической лаборатории является очень распространенным, поскольку именно с его помощью создается структура и содержание сайта.

\subsubsection{Язык программирования PHP}

PHP (рекурсивный акроним ``PHP: Hypertext Preprocessor'') - это очень популярный язык программирования с открытым исходным кодом, который особенно подходит для веб-разработки и может быть встроен в HTML. Синтаксис PHP похож на C, Java и Perl, и используется для создания динамических веб-приложений и интерактивных веб-сайтов.

PHP - это язык программирования на стороне сервера с открытым исходным кодом, который в основном используется для разработки динамических веб-приложений. Он был создан в 1994 году Расмусом Лердорфом как набор скриптов для отслеживания посетителей его личного сайта, и с тех пор стал одним из наиболее широко используемых языков программирования в Интернете.

PHP имеет C-подобный синтаксис и предназначен для взаимодействия с базами данных и доставки динамического содержимого через Интернет с помощью технологии веб-сервера. Некоторые из наиболее заметных особенностей PHP следующие:

\begin{itemize}
	\item  Интерпретация: исходный код PHP выполняется на сервере перед отправкой страницы в браузер пользователя.
	\item Простая интеграция с HTML: PHP позволяет внедрять PHP-код внутрь HTML-страниц для создания динамических веб-страниц.
	\item Подключение к базе данных: PHP имеет широкий спектр расширений для подключения к базам данных, что облегчает разработку приложений, использующих базы данных.
	\item Объектно-ориентированный: PHP также поддерживает объектно-ориентированное программирование.
	\item Обширная документация и сообщество: PHP имеет обширную документацию и активное онлайн-сообщество пользователей, которые предлагают поддержку и ресурсы для разработки на этой технологии.
\end{itemize}

PHP работает на сервере и генерирует HTML, который отправляется в браузер пользователя для отображения. Помимо HTML, PHP может генерировать и другие типы контента, такие как изображения и PDF-файлы\cite{PHP}.

Использование PHP на сайте клинической лаборатории полезно, поскольку это язык программирования, позволяющий создавать динамический контент на сайте. PHP использовался для создания интерактивных форм, автоматизации онлайн-задач, интеграции баз данных и защиты информации о пациентах.

В клинической лаборатории PHP может быть использован для создания страницы с интерактивными формами, позволяющими врачам регистрироваться и получать информацию в автоматическом режиме.

PHP имеет множество функций безопасности для защиты целостности данных пациентов, что делает его хорошим выбором для медицинской среды.

Его должны внедрять опытные программисты.

\subsubsection{Язык программирования CSS}

CSS (Cascading Style Sheets) - это язык программирования, используемый для определения визуального представления веб-страницы. С помощью CSS можно определить стиль элементов HTML, таких как текст, изображения, таблицы, формы и другие элементы страницы.

Правила CSS записываются в отдельном файле от HTML-файла и связываются с ним тегом <link> в заголовке HTML-документа. Правила CSS состоят из селектора и набора свойств и значений. Селектор указывает, какой элемент HTML должен быть стилизован, а свойства и значения определяют, как этот элемент должен быть отображен\cite{CSS}.

CSS также позволяет создавать сложные макеты с использованием таких приемов, как компоновка в виде коробки и сетки. Кроме того, можно использовать расширенные селекторы для применения стилей к определенным элементам на основе их положения на странице или их отношения к другим элементам.

Стили CSS записываются в отдельных файлах от HTML, но на них часто ссылаются из HTML-файлов с помощью специального тега <link>. Кроме того, стили могут применяться как к конкретным элементам, так и к классам, а также могут быть отзывчивыми, адаптируя макет страницы к различным размерам экрана\cite{css}.

Использование CSS на сайте клинической лаборатории необходимо для создания привлекательного и последовательного визуального оформления всего сайта.

На сайте клинической лаборатории CSS используется для применения единых стилей ко всем элементам страницы, что облегчает навигацию для врачей. Он также использовался для включения соответствующих цветовых схем и улучшения доступности контента. CSS использовался для создания отзывчивого дизайна, что означает, что веб-сайт будет адаптироваться к размеру экрана врача, улучшая его восприятие.

С помощью CSS сайт легко читается благодаря правильным приемам дизайна и соответствующему оформлению.

\paragraph{Преимущества использования языка программирования CSS}

Преимущества использования CSS (каскадных таблиц стилей) для разработки дизайна и стиля веб-сайтов следующие:

\begin{itemize}
	\item Разделение стиля и содержания: CSS позволяет нам отделить визуальный дизайн от содержания веб-сайта, что обеспечивает большую гибкость и простоту в изменении стиля.
	\item Меньше кода: использование CSS позволяет значительно сократить объем кода, необходимого для разработки дизайна и стиля веб-сайта, по сравнению с использованием встроенных стилей или атрибутов HTML.
	\item Точный контроль над внешним видом: CSS обеспечивает больший контроль над внешним видом элементов сайта, позволяя создавать более последовательные и профессиональные дизайны.
	\item Простота обслуживания: благодаря хранению кода CSS в отдельных файлах, обновления и изменения в дизайне могут быть сделаны более легко и быстро.
	\item Повышенная доступность: CSS позволяет создавать более доступные веб-сайты, поскольку визуальное представление может быть полностью отделено от содержания сайта.
	\item Согласованность дизайна: CSS позволяет создать централизованную таблицу стилей, гарантируя, что все страницы сайта будут иметь последовательный и хорошо организованный внешний вид.
\end{itemize}

\subsubsection{Система управления MySQL}

MySQL - это система управления реляционными базами данных (РСУБД) с открытым исходным кодом, основанная на языке структурированных запросов (SQL). Она разработана и распространяется корпорацией Oracle. MySQL используется в самых разных приложениях, от малого бизнеса до крупных корпораций, и хорошо интегрируется с популярными языками программирования, такими как PHP, Java и Python. MySQL поддерживает множество пользователей и различные платформы, включая Windows, Linux и macOS. Кроме того, он предлагает расширенные возможности, такие как транзакции ACID и поддержка хранимых процедур и триггеров. В целом, MySQL - это широко используемая и очень универсальная система управления реляционными базами данных, которая предлагает множество дополнительных функций для удовлетворения потребностей широкого круга пользователей\cite{MySQL}.

\paragraph{Преимущества использования MySQL}

Преимущества использования MySQL в качестве менеджера баз данных следующие:

\begin{itemize}
	\item Открытый исходный код: MySQL является открытым и бесплатным, что делает его доступным для малого бизнеса и стартапов.
	\item Масштабируемость: MySQL обладает высокой масштабируемостью и может обрабатывать большие объемы данных, что делает его подходящим для растущих компаний.
	\item Совместимость: MySQL совместим с большим количеством языков программирования, включая Python, Java, PHP и другие.
	\item Безопасность: MySQL имеет надежные средства защиты, включая аутентификацию пользователей и шифрование данных.
	\item Поддержка сообщества: MySQL имеет большое сообщество разработчиков и пользователей, готовых делиться информацией и оказывать поддержку по различным вопросам, связанным с базами данных.
	\item Надежность: MySQL является одной из старейших баз данных и постоянно тестируется и совершенствуется на протяжении многих лет, что гарантирует ее хорошую производительность.
\end{itemize}

\paragraph{Недостатки языка HTML}

К недостаткам языка HTML относятся:
\begin{itemize}
	\item Это статичный язык: HTML предназначен для статичных или нединамичных веб-страниц, что означает, что они не могут адаптироваться к потребностям и предпочтениям пользователей.
	\item Он не имеет четкой семантики: хотя HTML позволяет структурировать веб-страницу, нет четкого и определенного способа объяснить семантику содержимого. Хотя в HTML5 появилось больше семантических элементов, таких как <article>, <section>, <nav> и т.д.
	\item Интерпретация браузерами может отличаться: разные веб-браузеры могут по-разному интерпретировать одни и те же теги HTML, что может привести к тому, что внешний вид веб-страницы будет отличаться в каждом браузере.
	\item Некоторые теги могут устареть: HTML - постоянно развивающийся язык, и некоторые теги могут устареть после выхода новых версий и быть удалены.
	\item Сложность макета может увеличиться: по мере добавления новых элементов и определения большего количества стилей сложность макета страницы может увеличиться, что может усложнить процесс обслуживания.
\end{itemize}

Важно отметить, что, несмотря на эти недостатки, HTML является одним из самых популярных и широко используемых языков в веб-программировании благодаря своей простоте и универсальности\cite{Nhtml}.

\paragraph{Недостатки языка CSS}

К недостаткам CSS относятся:
\begin{itemize}
	\item Отсутствие обратной совместимости: Несмотря на усилия по поддержанию обратной совместимости, некоторые браузеры не поддерживают все возможности старых версий CSS или обрабатывают их по-другому, что может привести к нежелательному пользовательскому опыту.
	\item Кривая обучения: Изучение CSS поначалу может быть трудным. Вы должны изучить множество свойств и значений, чтобы добиться желаемого внешнего вида веб-страницы, а в некоторых случаях это может потребовать определенных художественных навыков в визуальном дизайне.
	\item Проблемы с производительностью: В некоторых случаях неправильная реализация CSS может негативно повлиять на производительность веб-страницы. CSS необходимо загрузить и разобрать, прежде чем браузер сможет загрузить содержимое. Кроме того, чрезмерная стилизация может негативно повлиять на скорость загрузки.
	\item Кроссбраузерные ограничения и несовместимость: Несмотря на стандартизацию органами спецификации, все еще существуют ограничения в функциональности некоторых свойств и значений, которые могут вызвать проблемы совместимости между различными браузерами, устройствами и операционными системами.
\end{itemize}

При разработке веб-сайта важно знать об этих недостатках, чтобы принимать обоснованные решения о том, как и когда использовать CSS\cite{Ncss}.

\paragraph{Недостатками MySQL}

\begin{itemize}
	\item Ограничения в масштабируемости: Хотя MySQL способен обрабатывать большие объемы данных, он может иметь ограничения в плане горизонтальной масштабируемости, что означает, что он может испытывать трудности с обработкой большого количества одновременных транзакций.
	\item Отсутствие поддержки некоторых расширенных возможностей: В отличие от других баз данных, MySQL не поддерживает некоторые расширенные возможности, такие как репликация в реальном времени и аварийное восстановление.
	\item Отсутствие поддержки некоторых аналитических функций: MySQL не имеет встроенной поддержки некоторых продвинутых аналитических функций, что может быть ограничением для некоторых приложений.
	\item Проблемы безопасности: Хотя MySQL в целом безопасна, могут существовать уязвимости, которые могут быть использованы хакерами, если не принять адекватных мер для защиты базы данных.
	\item Проблемы с производительностью: В некоторых случаях MySQL может иметь проблемы с производительностью, если он неправильно настроен или используется в среде с высокой нагрузкой.
\end{itemize}

\subsection{Диаграмма компонентов и обмен данными между пользователями и формами}

Диаграмма компонентов изображает связь каждой формы специальности с базой данных врачей и пациентов, как показано на рисунке\ref{fig:diag}.

\begin{figure}[H]
\center{\includegraphics[width=1\linewidth]{Diagrama de componentes}}
\caption{Диаграмма компонентов}
\label{fig:diag}
\end{figure}

Компоненты будут вызываться сценарием веб-страницы. На веб-странице будут отражены данные, которые необходимо ввести.

На рисунке \ref{fig:comp} представлена схема обмена данными между сценариями компонента при вызове компонента на странице сайта.

\begin{figure}[H]
\center{\includegraphics[width=1\linewidth]{Diagrama 2}}
\caption{Диаграмма компонентов}
\label{fig:comp}
\end{figure}

При вызове компонента сценарий веб-страницы указывает значения параметров компонента, которые затем передаются в файл index.html через логин проверки пользователя.

В сценарии файла index.html один из шаблонов компонента вызывается через метод POST к файлу admin, который отвечает за регистрацию информации в базе данных. В шаблоне medicos.php он вызывается из реестра медицинских пользователей, чтобы сгенерировать шаблон записи пациента.

В скрипте файла patient.php через метод POST к файлу admin вызывается один из шаблонов компонента, который отвечает за регистрацию информации в базе данных. В шаблоне пациента он вызывается из записи пациента по специальностям клинической лаборатории, и генерируется шаблон ввода данных пациента по выбранной специальности.

Работа компонента заканчивается, как только завершается сценарий файла analysis.php, т.е. можно выполнять действия после подключения файла шаблона.

\subsection{Диаграмма размещения}

Диаграмма размещения (рис.~\ref{place:image}) отражает физические взаимосвязи между программными и аппаратными компонентами системы. Она является хорошим средством для показа маршрутов перемещения объектов и компонентов в распределенной системе.

\begin{figure}[H]
\center{\includegraphics[width=0.37\linewidth]{place}}
\caption{Диаграмма размещения}
\label{place:image}
\end{figure}

\paragraph{Диаграмма таблицы MySQL}

Диаграммы таблиц в MySQL являются полезным инструментом для визуализации структуры базы данных и связей между ее таблицами, поскольку они облегчают понимание структуры базы данных. С помощью диаграммы таблиц можно наглядно и просто представить структуру базы данных MySQL и помочь в ее создании, имея глобальное видение структуры базы данных, легче ее спроектировать и создать подходящим образом.

Диаграмма на рисунке ~\ref{image:bas} показывает мертвые таблицы, созданные с помощью MySQL.

\begin{figure}[H]
	\center{\includegraphics[width=1\linewidth]{base.jpg}}
	\caption{Диаграмма на мертвые таблицы}
	\label{image:bas}
\end{figure}

\subsection{Содержание информационных блоков. Основные сущности}

Анализ требований выявляет три основные сущности:
\begin{itemize}
\item ``Врачи'';
\item ``Пациенты'';
\item ``Анализ''.
\end{itemize}

В состав сущности ``Врачи'' можно включить атрибуты, представленные в таблице \ref{tab:doctor}.

\begin{longtable}[l]{|C{7.5cm}|C{3.5cm}|C{3.5cm}|}
\caption{Атрибуты сущности ``Новости''\label{tab:doctor}}\\
\hline Поле & Тип & Обязательное \\
\hline 1 & 2 & 3 \\
\endfirsthead
\caption*{Продолжение таблицы \ref{tab:doctor}}\\
\hline 1 & 2 & 3 \\
\endhead
  \hline id & int & true \\
  \hline LastName & varchar & true \\
  \hline Name & varchar & true \\
  \hline Specialty & varchar & true \\
  \hline Officenumber & int & true \\
  \hline Nursing & varchar & false \\
  \hline
\end{longtable}

В состав сущности ``Пациенты'' включить атрибуты, представленные в таблице \ref{tab:patient}.
\begin{longtable}[l]{|C{7.5cm}|C{3.5cm}|C{3.5cm}|}
\caption{Атрибуты сущности ``Пациенты''\label{tab:patient}}\\
\hline Поле & Тип & Обязательное \\
\endfirsthead
\caption*{Продолжение таблицы \ref{tab:patient}}\\
\hline Поле & Тип & Обязательное \\
\endhead
  \hline id & int & true \\
  \hline Name & varchar & true \\
  \hline LastName & varchar & true \\
  \hline Gender & varchar & true \\
  \hline Birthday & date & true \\
  \hline Age & int & true \\
  \hline BloodType & varchar & true \\
  \hline Number & int & true \\
  \hline
\end{longtable}

В состав сущности ``Анализ'' включить атрибуты, представленные в таблице \ref{tab:analisis}.
\begin{longtable}[l]{|C{7.5cm}|C{3.5cm}|C{3.5cm}|}
	\caption{Атрибуты сущности ``Пациенты''\label{tab:analisis}}\\
	\hline Поле & Тип & Обязательное \\
	\endfirsthead
	\caption*{Продолжение таблицы \ref{tab:analisis}}\\
	\hline Поле & Тип & Обязательное \\
	\endhead
	\hline Floor & int & false \\
	\hline Officenumber & int & true \\
	\hline idPatient & int & true \\
	\hline Diagnosis & varchar & true \\
	\hline Prehistory & varchar & true \\
	\hline Processes & varchar & true \\
	\hline idDoctor & int & true \\
	\hline Diagnose & varchar & true \\
	\hline Treatment & varchar & true \\
	\hline Result & varchar & true \\
	\hline
\end{longtable}

Система имеет интегрированный механизм, соединяющий различные разделы и элементы информационных блоков, что означает отсутствие необходимости добавления дополнительных идентификаторов для связи между различными сущностями.

Информационные блоки содержат элементы, которые представляют сущности, а эти элементы, в свою очередь, имеют поля и свойства, которые представляют атрибуты этих сущностей. Таким образом, нет необходимости добавлять дополнительные идентификаторы для связи между различными частями системы.

\paragraph{Описание и разработка кода в таблицах для клинической лаборатории}

В таблицах, используемых для записи данных о пациентах, врачах и специальностях, в каждой строке требуются определенные элементы, некоторые из которых являются обязательными, а некоторые - необязательными. Каждая строка таблицы представляет собой запись данных и состоит из полей данных и ключей для установления связей между листами формы и агрегированными значениями. Кроме того, таблицы должны включать определенное хранилище, иметь ограничение на объем чтения и записи и, для обеспечения сохранности данных, иметь максимально допустимый размер символов для ввода и хранения.

Таблицы были созданы с помощью MySql, где определены типы данных и первичные ключи, используемые для каждой из таблиц, которые будут хранить информацию, введенную по правилам, установленным в каждом текстовом поле формы.

Для всех полей в медицинской карте пациента, врача и специальности требуемая информация должна быть заполнена так, как указано в поле, приложение выдаст ошибку, если пользователь введет недопустимый символ между цифрами и буквами. Таблицы были созданы и предназначены для создания отчетов для соответствующей поддержки.

\subsection{Словарь тематического охвата продуктов веб-приложений}

На основании анализа предметной области из пункта 1 технического задания был составлен словарь предметной области. В словаре представлены термины на aнглийский и испанский языках.

В таблице ~\ref{tab:login} приведен словарь данных разрабатываемой среды веб-приложения, которая отвечает за проверку пользователей и паролей.

\begin{longtable}[l]{|C{6.5cm}|C{8.5cm}|}
	\caption{Пользователей и паролей\label{tab:login}}\\
	\hline Термин & Описание\\
	\endfirsthead
	\caption*{Продолжение таблицы \ref{tab:login}}\\
	\hline Термин & Описание\\
	\endhead
	\hline Поле ``пользователь врача''. & Поле, в котором необходимо ввести пользователя врача, заранее зарегистрированного в базе данных\\
	\hline Поле ``пароль врача''. & Поле, в котором необходимо ввести пароль врача, заранее зарегистрированного в базе данных.\\
	\hline кнопка валидации & При нажатии кнопки система считывает и проверяет информацию, введенную в поле имени пользователя и пароля\\
	\hline
\end{longtable}

На рисунке \ref{image:err} ниже показано сообщение об ошибке при неправильном вводе имени пользователя и пароля врача.

\begin{figure}[H]
	\center{\includegraphics[width=1\linewidth]{error.png}}
	\caption{Сообщение об ошибке при вводе неправильного имени пользователя и пароля}
	\label{image:err}
\end{figure}

В окне ошибки проверки пользователя также будет кнопка ``HOME'', которая перенаправит обратно в главное окно для ввода имени пользователя и пароля. Формы, представленные в шаблонах, также соответствуют строгим правилам заполнения обязательных полей, чтобы избежать пустых мест в базе данных и не создавать проблем при вводе или регистрации новой информации в базе данных.
