\newsection
\setcounter{figure}{0}\setcounter{table}{0}
Структура и объем работы. Отчет состоит из введения, 4 разделов основной части, заключения, списка использованных источников, 2 приложений. Текст выпускной квалификационной работы включает 53 страницы, из них 40 страниц основного текста\section{Анализ предметной области}
\subsection{Характеристики компании и пользователей}

Клиническая лаборатория - это место, где проводятся различные виды медицинских анализов и тестов, которое характеризуется наличием специализированной техники и инструментов. К общим характеристикам клинической лаборатории относятся:

\begin{itemize}
	\item Она имеет различные специализации: клиническая биохимия, бактериология, гематология, иммунология и микробиология, среди прочих.
	\item Она отвечает за диагностические тесты и медицинские исследования.
	\item Услуги, предлагаемые лабораторией, - это определение in vitro биологических свойств человека (или животного, в ветеринарии).
	\item Интерпретация полученных результатов осуществляется специализированным медицинским работником.
	\item В клинической лаборатории предусмотрены меры контроля качества и безопасности для обеспечения точности и надежности результатов.
	\item Для проведения анализов обычно используются кровь, моча, фекалии, ткани и другие виды биологических образцов.
\end{itemize}

В целом, характеристики клинической лаборатории могут варьироваться в зависимости от типа проводимого анализа и преследуемых конкретных целей.

\subsection{HTML-программирование}

Язык гипертекстовой разметки HTML был разработан британским ученым Тимом Бернерсом-Ли примерно в 1986-1991 годах в ЦЕРНе в Женеве, Швейцария. HTML был создан как язык для обмена научной и технической документацией, пригодный для использования неспециалистами в области верстки. HTML успешно справился со сложностью SGML, определив небольшой набор структурных и семантических элементов, называемых дескрипторами. Дескрипторы также часто называют ``тегами''. С помощью HTML можно легко создать относительно простой, но красиво оформленный документ. Помимо упрощения структуры документа, в HTML была добавлена поддержка гипертекста. Мультимедийные функции были добавлены позже.

HTML - это язык разметки, используемый для определения содержания веб-страниц. Он составлен на основе тегов, также называемых метками или тегами, с помощью которых мы выражаем части документа, заголовок, тело, заголовки, параграфы и т.д. Короче говоря, содержание веб-страницы\cite{Html}.

Клиника ``KLLaboratory'' основана на использовании веб-страницы с HTML для ввода личных данных врачей и пациентов, она также помогает в цифровой регистрации обследований, проведенных каждому пациенту, и результатов этих обследований с помощью форм.Интерфейс сайта основан на наборе анимаций для удобства работы и поиска врачей, пациентов и специальностей.

Цель веб-сайта клиники - ускорить и гарантировать регистрацию всех данных, введенных обученным персоналом в предоставленные формы. С помощью языка HTML и его компонентов был создан основной шаблон для входа администратора и врача-пользователя, а в качестве вторичного шаблона множественный вариант для регистрации пациентов и специальностей. 
Клиника будет открыта для всех видов пациентов, включая несовершеннолетних и их регистрацию.

\paragraph{Преимущества использования HTML}

Преимуществами использования HTML (HyperText Markup Language) для создания веб-сайтов являются:

\begin{itemize}
	\item Четкая структура сайта: HTML обеспечивает четкую и организованную структуру содержимого сайта, что облегчает его понимание и навигацию.
	\item Простота изучения: HTML - это простой и понятный язык разметки, что делает его доступным для тех, кто только начинает создавать веб-сайты.
	\item Совместимость с различными платформами: HTML совместим с различными веб-браузерами и операционными системами, что позволяет просматривать веб-сайт на разных платформах.
	\item Высокая функциональная совместимость: HTML может работать в сочетании с другими языками программирования и технологиями, например, в сочетании с CSS для дизайна и оформления сайта.
	\item Большое сообщество пользователей: HTML - очень популярный язык для создания веб-сайтов, поэтому он имеет большое сообщество разработчиков и пользователей, готовых делиться информацией и оказывать поддержку.
\end{itemize}

Панель навигации основного шаблона содержит:

\begin{itemize}
	\item Пациенты;
	\item Анализы;
	\item Врачи.
\end{itemize}

Программное обеспечение, разработанное специально для использования в клинических лабораториях, предоставляет ряд инструментов, повышающих эффективность и точность работы. Эти программы могут предложить отслеживание образцов, управление запасами, планирование анализов, управление пациентами и инструменты для соблюдения нормативных стандартов. Эти программы также могут улучшить коммуникацию, управление назначением и доступ пациентов к результатам анализов более легко и быстро. Одним словом, использование программного обеспечения в клинической лаборатории повышает эффективность, точность, управление и удобство для пациентов\cite{html}.

\subsection{Преимущества веб-сайта в клинике}

Использование программного обеспечения в клинической лаборатории может обеспечить ряд преимуществ, некоторые из которых включают:
\begin{itemize}
	\item повышение эффективности: программное обеспечение может автоматизировать многие рутинные задачи, которые лаборанты часто выполняют вручную, такие как ввод данных, маркировка образцов, проверка результатов и создание отчетов. Это не только повышает эффективность, но и уменьшает количество ошибок и снижает необходимость в контроле;
	\item более эффективное управление данными: Клинические лаборатории обрабатывают и хранят большие объемы данных о пациентах, включая результаты анализов, протоколы лечения и медицинские карты. Программные системы могут помочь более эффективно собирать и хранить эти данные, облегчая выявление закономерностей и тенденций, создание специализированных отчетов и принятие обоснованных решений;
	\item повышение качества работы: использование программного обеспечения может помочь уменьшить количество ошибок и повысить объективность результатов. Например, некоторые системы используют автоматизированные тесты для получения более точных и последовательных результатов, что повышает качество работы в лаборатории. Программное обеспечение также может помочь стандартизировать процедуры тестирования, уменьшить вариации и улучшить воспроизводимость результатов;
	\item  соответствие нормативным требованиям: Клинические лаборатории подчиняются многочисленным нормам и стандартам, и использование специализированного программного обеспечения может помочь обеспечить соответствие этим нормам. Например, некоторые системы могут контролировать доступ к данным пациента и вести учет всех проведенных тестов и полученных результатов.
\end{itemize}

\subsection{Требования пользователей к продукту веб-приложения}

Веб-сайт может быть инструментом информирования и продвижения услуг клинической лаборатории, позволяя врачам быстро найти подробную информацию о предлагаемых услугах, результатах, диагнозах и другую важную информацию.

Веб-сайт также можно использовать для записи важных и актуальных данных для последующего наблюдения за приемом лекарств или лечением пациента.

Веб-сайт в клинической лаборатории помогает максимально отказаться от использования бумажных листов и положиться на виртуальную систему с гораздо большим объемом памяти для записи всего, что запрашивают врачи относительно медицинских обследований, проводимых пациентам каждой специальности.

Пользовательские требования к продукту веб-приложения относятся к потребностям и ожиданиям конечных пользователей продукта. Чтобы определить эти требования, важно провести исследование пользователей и проанализировать следующие аспекты:

\begin{itemize}
	\item Функциональность: какими возможностями должен обладать продукт, чтобы удовлетворить потребности пользователей. Например, если это приложение для онлайн-покупок, пользователям может потребоваться система корзины, расширенные возможности поиска, интегрированная платежная система и т.д.
	\item Удобство использования: как должен быть спроектирован пользовательский интерфейс, чтобы продукт был прост в использовании и понятен пользователям. Необходимо изучить предпочтения пользователей в использовании, способность легко выполнять задачи, доступность, обратную связь и способность узнавать о функциях продукта.
	\item Пользовательский опыт: как продукт ощущается и как он отвечает ожиданиям пользователей. Сюда входят такие аспекты, как эстетика и визуальный дизайн, качество контента и функций, чувство удовлетворения и возможность получить удовольствие от пользовательского опыта.
	\item Производительность и безопасность: какие требования к безопасности и производительности должны быть выполнены для обеспечения оптимального уровня безопасности и скорости работы продукта. Сюда входят такие аспекты, как отзывчивость на различные устройства, надежность в работе, масштабируемость и безопасность данных.
\end{itemize}

\paragraph{Исследование предметной области}

Веб-приложение - это язык, предоставляющий определенный перечень команд, с помощью которых пользователь может взаимодействовать с веб-сайтом клиники. 
Для того чтобы получить доступ к различным опциям на сайте, необходимо зарегистрироваться, введя имя пользователя и пароль.
На сайте есть кнопки для сохранения и выбора различных форм.
В основном интерфейсе пользователь сможет ввести имена пользователей и пароли для проверки системой.