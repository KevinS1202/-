\newsection
\section{Рабочий проект}
\subsection{Модули, используемые при разработке сайта}

Можно выделить следующий список модулей и их функций, которые были использованы при разработке веб-приложения (таблица \ref{table:metod}).

\begin{longtable}[l]{|p{3cm}|p{3cm}|p{4cm}|p{4cm}|}
\caption{Описание модулей, используемых в приложении\label{table:metod}}\\
\hline Название модуля & Функция, к которой принадлежит модуль & Описание модуля & Функции \\
\hline \centering 1 & \centering 2 & \centering 3 & 4\\
\endfirsthead
\caption*{Продолжение таблицы \ref{table:metod}}\\
\hline \centering 1 & \centering 2 & \centering 3 & 4\\
\endhead
\hline Conexion.php & Подключение к базе данных & Функция ``pdo'' используется для вызова базы данных и выполнения аутентификации MySQL & pdo=new PDO(mysql:host={
config[host]};
dbname={config
[dbname]},
config[user],
config[password])
проверьте правильность подключения к базе данных
\cr
\hline Save.php & Функция помогает записывать данные в таблицы базы данных & Вызывается функция подключения, чтобы проверить данные в базе данных и иметь возможность ввести информацию в выбранную таблицу & С функцией conn->prepare('INSERT INTO patients (Name, LastName, Gender, Birthday, Age, Bloodtype, Number) VALUES (:name, :lastname, :gender, :birthday, :age, :bloodtype, :number)');
\cr
\hline formulario.html & Метод ``Post'' используется для вызова модуля Save.php & Модуль формы представляет собой форму для заполнения полей специальностей клинической лаборатории & Функция использует названные выше поля с ``label'' в текстовом формате каждое, чтобы извлечь информацию, введенную пользователями
\cr
\hline nav.html & Это простое в навигации меню для выбора пациентов или специальностей & Меню отображается в левой части экрана с указанными опциями для удобства использования в клинической лаборатории & С помощью функции ``href'' из меню выбирается указанный путь выбранной формы, каждая ссылка содержит свое имя для различения
\cr
\hline login.php & Помощь в проверке достоверности данных для зарегистрированного пользователя, вошедшего в систему и получившего право манипулировать данными & Функция ``session start()'' вызывается для проведения соответствующей проверки имени пользователя и пароля врачей, которым разрешено пользоваться системой & Для проверки с базой данных зарегистрированных медицинских пользователей используется условие, функция такова: if (!empty( POST['username']) BB!empty( POST['password']) {require'../admin/
Conexion.php'
conn=conexion
(config)}
\cr
\hline Patient.php & Это форма для ввода личных данных пациентов & Когда вы заполните все данные пациента, необходимо нажать кнопку ``Сохранить'', после чего будет вызван модуль ``Save.php'' для проверки информации и последующей регистрации в базе данных & Форма пациента работает с различными типами ``меток'' с соответствующими названиями и типом формата ввода, от пользователя для считывания и регистрации в базе данных
\cr
\hline doctors.php & Это форма для ввода личных данных врачей, а также имен пользователей и паролей для входа в систему & Когда вы заполните все данные о враче, нажмите на ``Сохранить'', и будет вызван модуль ``Save.php'' для проверки информации и последующей регистрации в базе данных. & Форма врача работает с различными типами ``тегов'' с соответствующими именами и типом формата ввода, для чтения и записи в базу данных пользователем
\cr
\hline
\end{longtable}

\subsection{Тестирование разработанного web-сайта}

На рисунке \ref{imag:pri}показана страница регистрации пациента на сайте клинической лаборатории.

\begin{figure}[H]
\center{\includegraphics[width=1\linewidth]{pag1.jpg}}
\caption{Главная страница сайта клинической лаборатории}
\label{imag:pri}
\end{figure}

На рисунке \ref{image:esp}показана страница каждой специальности на сайте клинической лаборатории.

\begin{figure}[H]
\center{\includegraphics[width=1\linewidth]{pag 2.jpg}}
\caption{Страница каждой специальности}
\label{image:esp}
\end{figure}

На рисунке \ref{image:doc}показана страница регистрации данных врача на сайте клинической лаборатории.

\begin{figure}[H]
\center{\includegraphics[width=1\linewidth]{pag3.jpg}}
\caption{Страница регистрации данных врача}
\label{image:doc}
\end{figure}
