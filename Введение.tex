\newpage
\begin{center}ВВЕДЕНИЕ\end{center}

  \addcontentsline{toc}{section}{ВВЕДЕНИЕ}

Сегодня подавляющее большинство медицинских лабораторий продолжают использовать бумажные документы для хранения информации о пациентах. Однако этот метод может привести к ошибкам при наборе текста и трудностям при передаче и чтении информации. Пациентам и врачам приходится проходить через клинику и физическую лабораторию для получения результатов, что может задерживать лечение пациента. Для решения этой проблемы было проведено исследование, в ходе которого было предложено решение для облегчения управления процессами и процедурами с использованием современных технологий, например, создание приложения для хранения и редактирования информации, вводимой врачами.

Предложенное решение позволит ускорить время и снизить процент ошибок при наборе текста и потерю физических документов. Для достижения этой цели будет поощряться использование компьютера и подключения к Интернету. Работа будет включать разработку и внедрение алгоритма, связанного с формированием и получением информации. Кроме того, будет разработан алгоритм поиска для каждого лабораторного отделения, входящего в состав клиники.

В целом, цель данной работы - обеспечить решение проблемы ведения медицинской документации, используя современные технологии для облегчения процессов и процедур, а также сокращения ошибок и потери информации.

Целью данной работы является разработка сайта для клинической лаборатории компании ``KLLaboratory'', для облегчения и ускорения регистрации информации. Для достижения поставленной цели необходимо решить следующие задачи:
\begin{itemize}
\item провести анализ предметной области;
\item разработать структуру web-сайта;
\item реализовать разработанную структуру средствами web-технологий с \linebreak применением CMS.
\end{itemize}

Структура и объем работы. Отчет состоит из введения, 4 разделов основной части, заключения, списка использованных источников, 2 приложений. Текст выпускной квалификационной работы включает 53 страницы, из них 40 страниц основного текста.

В первом разделе, на этапе описания технических характеристик тематического участка, содержится сбор информации о деятельности клинической лаборатории по его разработке.

Второй раздел, на этапе технического задания, содержит требования к разрабатываемому сайту.

Третий раздел, на этапе технического проектирования, содержит дизайнерские решения для сайта.

В четвертом разделе приводится перечень классов и их методов, использованных при разработке сайта, проводится тестирование разработанного сайта.

В заключении представлены основные результаты работы, полученные в ходе разработки.

Приложение А содержит графический материал.

Приложение Б содержит текст макета сайта, подключение компонентов. 
